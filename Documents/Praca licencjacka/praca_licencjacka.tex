%
% Niniejszy plik stanowi przyk�ad formatowania pracy magisterskiej na
% Wydziale MIM UW.  Szkielet u�ytych polece� mo�na wykorzystywa� do
% woli, np. formatujac wlasna prace.
%
% Zawartosc merytoryczna stanowi oryginalnosiagniecie
% naukowosciowe Marcina Wolinskiego.  Wszelkie prawa zastrze�one.
%
% Copyright (c) 2001 by Marcin Woli�ski <M.Wolinski@gust.org.pl>
% Poprawki spowodowane zmianami przepis�w - Marcin Szczuka, 1.10.2004
% Poprawki spowodowane zmianami przepisow i ujednolicenie 
% - Seweryn Kar�owicz, 05.05.2006
% dodaj opcj� [licencjacka] dla pracy licencjackiej
\documentclass{pracamgr}

\usepackage{polski}

%Jesli uzywasz kodowania polskich znakow ISO-8859-2 nastepna linia powinna byc 
%odkomentowana
%\usepackage[latin2]{inputenc}

\usepackage[utf8]{inputenc}
%Jesli uzywasz kodowania polskich znakow CP-1250 to ta linia powinna byc 
%odkomentowana
%\usepackage[cp1250]{inputenc}

% Dane magistranta:

\author{Tomasz Jurkowski\\ 
Cezary Lasocki\\ 
Małgorzata Salak\\
Krzysztof Wytrykowski}

\nralbumu{320682, 320813, 305549, 306481}

\title{Projekt Clarifier}

\tytulang{The Clarifier project}

%kierunek: Matematyka, Informatyka, ...
\kierunek{Informatyka}

% informatyka - nie okreslamy zakresu (opcja zakomentowana)
% matematyka - zakres moze pozostac nieokreslony,
% a jesli ma byc okreslony dla pracy mgr,
% to przyjmuje jedna z wartosci:
% {metod matematycznych w finansach}
% {metod matematycznych w ubezpieczeniach}
% {matematyki stosowanej}
% {nauczania matematyki}
% Dla pracy licencjackiej mamy natomiast
% mozliwosc wpisania takiej wartosci zakresu:
% {Jednoczesnych Studiow Ekonomiczno--Matematycznych}

% \zakres{Tu wpisac, jesli trzeba, jedna z opcji podanych wyzej}

% Praca wykonana pod kierunkiem:
% (poda� tytu�/stopie� imi� i nazwisko opiekuna
% Instytut
% ew. Wydzia� ew. Uczelnia (je�eli nie MIM UW))
\opiekun{dra Jacka Sroki\\
  Instytut Informatyki\\
  }

% miesi�c i~rok:
\date{Czerwiec 2014}

%Poda� dziedzin� wg klasyfikacji Socrates-Erasmus:
\dziedzina{ 
%11.0 Matematyka, Informatyka:\\ 
%11.1 Matematyka\\ 
%11.2 Statystyka\\ 
11.3 Informatyka\\ 
%11.4 Sztuczna inteligencja\\ 
%11.5 Nauki aktuarialne\\
%11.9 Inne nauki matematyczne i informatyczne
}

%Klasyfikacja tematyczna wedlug AMS (matematyka) lub ACM (informatyka)
\klasyfikacja{D.1.5 Object-oriented Programming\\
  D.2.6 Programming Environments\\}

% S�owa kluczowe:
\keywords{clarifier, nauka, student, wykładowca, zapamiętywanie, aplikacja mobilna, ASP.NET, azure, windows phone}

% Tu jest dobre miejsce na Twoje w�asne makra i~�rodowiska:
\newtheorem{defi}{Definicja}[section]

% koniec definicji

\begin{document}
\maketitle

%tu idzie streszczenie na strone poczatkowa
\begin{abstract}
W pracy opisano realizację projektu Clarifier. Jest to system wspomagający studentów w utrwalaniu materiałów objętych programami nauczania na uczelniach wyższych. System umożliwia wybór przedmiotów i zagadnień do utrwalania oraz oferuje różne tryby nauki. Ponadto, Clarifer pozwala wykładowcom na dodawanie własnych zadań, a także dostarcza dane na temat wyników uzyskiwanych przez studentów. W niniejszej pracy opisujemy motywację, architekturę, proces tworzenia oraz funkcjonalność systemu Clarifier. 
\end{abstract}

\tableofcontents
%\listoffigures
%\listoftables

\chapter*{Wprowadzenie}
\addcontentsline{toc}{chapter}{Wprowadzenie}

\section{Wstęp}

Wiedza zdobywana przez studentów w trakcie studiów niezwykle często jest przez nich zapominana wkrótce po zaliczeniu egzaminów. Większość studentów uczy się z własnych notatek lub materiałów dostarczanych przez
wykładowców. Materiały te zazwyczaj ukierunkowane są na krótkotrwałe zapamiętywanie zagadnień o dużym stopniu szczegółowości, a ich forma nie zachęca do utrwalania
wiedzy z danego przedmiotu po zdaniu egzaminu. Również samo zaliczenie przedmiotu poprzez zapamiętywanie pojęć bez pełnego zrozumienia ich sensu i zastosowań nie prowadzi do podniesienia swoich kwalifikacji.
\par W praktyce trwała i dogłębna znajomość pewnych zagadnień i umiejętność rozwiązywania określonych typów zadań jest niezbędna w takich sytuacjach jak rozmowy kwalifikacyjne czy egzaminy licencjackie i magisterskie, które w dzisiejszych czasach dotyczą większości studentów i absolwentów uczelni wyższych. Stąd istnieje potrzeba utrwalania i pogłębiania wiedzy zdobywanej w
trakcie studiów. Istotnym problemem, z jakim boryka się wielu studentów, którzy korzystają z systemów i aplikacji wspomagających naukę, jest brak informacji zwrotnej o aktualnym stopniu opanowania danego przedmiotu lub zagadnienia. Także wykładowcy nie mają zwykle dostępu do informacji o postępach swoich studentów i w efekcie nie są w stanie dostosować treści przekazywanych na zajęciach do wiedzy i umiejętności uczestników.  
\par Celem naszego projektu jest stworzenie systemu wspomagającego utrawalanie wiedzy zdobywanej na studiach, który jednocześnie dostarcza studentom i wykładowcom informacje o osiąganych wynikach. Niniejszy tekst jest dokumentacją naszej pracy włożonej w realizację tego systemu.

\section{Definicje}

Podstawowe pojęcia używane w niniejszej pracy:

\chapter{Wizja systemu}\label{r:vision}

\chapter{Opis architektury}\label{r:architecture}

\chapter{Organizacja pracy}\label{r:org}

\chapter{Dokumentacja użytkowa}\label{r:documentation}

\chapter{Podsumowanie}

\appendix

\chapter{Zawartość dysku CD}

\chapter{Podział pracy}

\begin{thebibliography}{99}
\addcontentsline{toc}{chapter}{Bibliografia}



\end{thebibliography}

\end{document}


%%% Local Variables:
%%% mode: latex
%%% TeX-master: t
%%% coding: latin-2
%%% End:
